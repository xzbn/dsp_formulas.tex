\documentclass[12pt,a4paper]{article}
\usepackage[margin=1in]{geometry}
\usepackage{amsmath,amssymb,amsfonts}
\usepackage{CJKutf8}
\usepackage{setspace}

\title{数字信号处理公式汇总 (DTFT, DFT, Z-Transform, Filter)}
\author{}
\date{}

\begin{document}
\begin{CJK}{UTF8}{gbsn}
\maketitle
\onehalfspacing

本文档汇总了数字信号处理 (DSP) 中与 DTFT、DFT、Z-变换以及滤波器设计相关的经典公式与常用表达式,以供复习与参考。

\section{离散时间傅里叶变换 (DTFT)}

\subsection{定义与逆变换}
\[
X(e^{j\omega}) = \sum_{n=-\infty}^{\infty} x[n] e^{-j \omega n}
\]
\[
x[n] = \frac{1}{2\pi}\int_{-\pi}^{\pi} X(e^{j\omega}) e^{j\omega n} d\omega
\]

\subsection{DTFT 重要性质}
\begin{enumerate}
\item 线性性:  
若 \(y[n] = a x_1[n] + b x_2[n]\),则  
\[
Y(e^{j\omega}) = a X_1(e^{j\omega}) + b X_2(e^{j\omega})
\]

\item 时间移位:  
\[
x[n-n_0] \xleftrightarrow{\text{DTFT}} e^{-j\omega n_0} X(e^{j\omega})
\]

\item 频率移位:  
\[
e^{j\omega_0 n}x[n] \xleftrightarrow{\text{DTFT}} X(e^{j(\omega-\omega_0)})
\]

\item 卷积与频域相乘:  
\((x_1 * x_2)[n] \xleftrightarrow{\text{DTFT}} X_1(e^{j\omega})X_2(e^{j\omega})\)

\item Parseval 定理:  
\[
\sum_{n=-\infty}^{\infty} |x[n]|^2 = \frac{1}{2\pi}\int_{-\pi}^{\pi} |X(e^{j\omega})|^2 d\omega
\]
\end{enumerate}

\subsection{常见序列的 DTFT}
\[
\delta[n] \xleftrightarrow{\text{DTFT}} 1
\]
\[
u[n] \xleftrightarrow{\text{DTFT}} \frac{1}{1 - e^{-j\omega}} \ (|\omega|\neq 0)
\]
\[
a^n u[n] \xleftrightarrow{\text{DTFT}} \frac{1}{1 - a e^{-j\omega}},\ |a|<1
\]

\section{离散傅里叶变换 (DFT)}

\subsection{定义与逆变换}
DFT 定义:
\[
X[k] = \sum_{n=0}^{N-1} x[n] e^{-j\frac{2\pi}{N}nk}, \quad k=0,\ldots,N-1
\]

IDFT 定义:
\[
x[n] = \frac{1}{N}\sum_{k=0}^{N-1} X[k] e^{j\frac{2\pi}{N}nk}, \quad n=0,\ldots,N-1
\]

\subsection{DFT 重要性质}
\begin{enumerate}
\item 线性性:
\[
a x_1[n] + b x_2[n] \xleftrightarrow{\text{DFT}} a X_1[k] + b X_2[k]
\]

\item 循环卷积:
若 \(y[n]\) 为长度 N 的循环卷积 \((x_1 \circledast x_2)[n]\),则
\[
Y[k] = X_1[k] X_2[k]
\]

\item Parseval 定理:
\[
\sum_{n=0}^{N-1} |x[n]|^2 = \frac{1}{N}\sum_{k=0}^{N-1} |X[k]|^2
\]

\item DFT 与 DTFT 的关系:DFT 是 DTFT 在频率上的抽样(针对有限长序列)。
\end{enumerate}

\section{Z-变换 (Z-Transform)}

\subsection{定义与逆变换}
\[
X(z) = \sum_{n=-\infty}^{\infty} x[n] z^{-n}, \quad z=re^{j\theta}
\]

逆 Z-变换可通过部分分式展开、查表或留数定理求得。

\subsection{常用 Z-变换对照表}
\[
\delta[n] \xleftrightarrow{Z} 1
\]
\[
u[n] \xleftrightarrow{Z} \frac{1}{1-z^{-1}},\ |z|>1
\]
\[
a^n u[n] \xleftrightarrow{Z} \frac{1}{1 - a z^{-1}},\ |z|>|a|
\]

\subsection{Z-变换性质}
\begin{enumerate}
\item 线性性:
\[
a x_1[n] + b x_2[n] \xleftrightarrow{Z} a X_1(z) + b X_2(z)
\]

\item 时间移位:
\[
x[n-n_0]u[n-n_0] \xleftrightarrow{Z} z^{-n_0} X(z)
\]

\item 卷积:
\[
(x_1 * x_2)[n] \xleftrightarrow{Z} X_1(z)X_2(z)
\]

\item 与 DTFT 的关系:当 \(z=e^{j\omega}\) 时,Z-变换在单位圆上即为 DTFT。
\end{enumerate}

\section{滤波器 (Filter)}

\subsection{系统函数与滤波器类型}
给定 LTI 系统差分方程:
\[
\sum_{k=0}^{M} a_k y[n-k] = \sum_{l=0}^{N} b_l x[n-l]
\]
Z 域:
\[
H(z) = \frac{Y(z)}{X(z)} = \frac{\sum_{l=0}^{N} b_l z^{-l}}{\sum_{k=0}^{M} a_k z^{-k}}
\]

\subsection{FIR 滤波器}
FIR 滤波器脉冲响应 \(h[n]\) 有限长,频率响应:
\[
H(e^{j\omega}) = \sum_{n=0}^{N-1} h[n] e^{-j\omega n}
\]

典型设计方法:窗口法设计低通滤波器时,理想低通滤波器频率响应为:
\[
H_d(e^{j\omega}) =
\begin{cases}
1, & |\omega| < \omega_c \\
0, & |\omega| > \omega_c
\end{cases}
\]
利用理想低通滤波器的脉冲响应 (sinc 函数) 截断并乘以窗函数获得 \(h[n]\)。

\subsection{IIR 滤波器}
IIR 滤波器通过模拟滤波器变换获得。如双线性变换:
\[
s = \frac{2}{T}\frac{z-1}{z+1}
\]
从模拟域巴特沃斯、切比雪夫滤波器传递函数 \(H_a(s)\) 替换 \(s\) 得到数字 \(H(z)\)。

\section{附加公式与概念}

\subsection{双线性变换}
\[
z = \frac{1 + \frac{sT}{2}}{1 - \frac{sT}{2}}, \quad s = \frac{2}{T}\frac{z-1}{z+1}
\]

\subsection{离散卷积定义}
\[
y[n] = (x*h)[n] = \sum_{m=-\infty}^{\infty} x[m] h[n-m]
\]

\subsection{相关函数定义}
自相关与互相关在 DSP 中用于分析信号特性:
\[
r_{xx}[\tau] = \sum_{n=-\infty}^{\infty} x[n]x[n-\tau]^*
\]

以上即为数字信号处理 (DTFT, DFT, Z-变换, Filter) 的主要公式汇总。

\end{CJK}
\end{document}
